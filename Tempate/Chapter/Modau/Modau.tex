\chapter* {MỞ ĐẦU}
\addcontentsline{toc}{chapter}{Mở đầu}

\tocless\section {Lí do chọn đề tài}
Cùng với sự phát triển của Internet, các mạng xã hội đã phát triển mạnh mẽ và trở thành một xu hướng mới thu hút nhiều người sử dụng trên toàn thế giới. Theo những khảo sát gần đây, có gần một nửa dân số thế giới, tức là hơn 3 tỷ người sử dụng mạng xã hội. Nhờ có mạng xã hội, người dùng có thể trao đổi thông tin với nhau một cách nhanh chóng bất kể khoảng cách về địa lý và thời gian. Bên cạnh đó, mạng xã hội còn cung cấp cho người dùng rất nhiều ứng dụng hữu ích, làm cho cuộc sống của con người ngày càng trở nên thuận tiện hơn. Ngoài những đặc tính kế thừa của mạng lưới xã hội thực như: Tương tác giữa người dùng, lan truyền thông tin, tạo ảnh hưởng trong cộng đồng thì mạng xã hội còn mang nhiều đặc tính mới như: Cập nhật thông tin thực lên mạng xã hội một cách nhanh chóng, sự lan truyền thông tin giữa người dùng xảy ra trong thời gian ngắn, sự bùng nổ thông tin với các nguồn tin tức khác nhau, v.v... Có thể nói, hiện nay mạng xã hội đang từng bước trở thành một kho tri thức mới mà con người có thể dễ dàng tiếp cận 

Tuy vậy, đi cùng với những lợi ích mạng xã hội mang lại thì còn nhiều nguy cơ và hiểm họa tới người dùng, trong đó có các yếu tố xấu như tin đồn sai lệch được lây lan nhanh chóng. Các yếu tố này gây ra những tác hại lớn đối với cộng đồng người sử dụng mạng xã hội. Không chỉ ở Việt Nam, những tác hại này diễn ra trên phạm vi toàn thế giới. Việc lan truyền thông tin sai lệch có thể gây ra những tác động xã hội tiêu cực, thậm chí là tổn thất lớn về kinh tế. Chẳng hạn, những tin đồn không hay về sức khỏe của tổng thống Mỹ ông Obama vào tháng 4 năm 2013 đã dẫn đến sự bất ổn của thị trường tài chính ở phố Wall. Trước thềm Đại hội đại biểu toàn quốc lần thứ XII của Đảng Cộng sản Việt Nam, đã có những thông tin về gia đình cũng như tài sản của Thủ tướng Nguyễn Xuân Phúc lúc đó vẫn còn là Phó Thủ tướng Chính phủ, nhằm hạ uy tín gây mất lòng tin của nhân dân đối với Thủ tướng.

Twitter, Youtube và Facebook đang phải đối mặt với việc gia tăng áp lực từ chính phủ của nhiều nước đe dọa áp đặt các đạo luật mới và khoản phạt nếu các mạng xã hội này không có biện pháp loại bỏ nhanh chóng các nội dung tuyên truyền cực đoan cũng như các nội dung vi phạm luật pháp các nước sở tại. 

Trong thời gian gần đây, Đảng và Nhà nước ta đã đề ra rất nhiều biện pháp nhằm ngăn chặn thông tin sai lệch trên mạng xã hội và đặc biệt là bộ Thông tin và truyền thông đã tiến hành ngăn chặn, xử lý những thông tin sai sự thật, nghiêm túc xem xét đánh giá, xử lý sai phạm, bộ Thông tin và truyền thông đã bắt các nhà MXH gỡ bỏ các thông tin sai lệch được đăng tải \cite{thongtin}. Bộ quan tâm trước hết đến báo chí và chất lượng báo chí, có những bài viết với số lượng nhiều nhưng chất lượng kém. Bên cạnh đó, ta chủ động phát hiện những thông tin sai lệch, kịp thời chỉnh sửa, đính chính, đưa ra thông tin chính xác nhất. 

Những thông tin sai lệch, không kiểm chứng, giật tít tạo sự tò mò cho người dùng trên mạng xã hội còn tiềm ẩn các nguy cơ phát tán mã độc, lừa đảo trên mạng gây mất an ninh, an toàn thông tin. Với thực trạng đó, các nhà khoa học đã nghiên cứu những giải pháp hiệu quả để ngăn chặn thông tin sai lệch. Trong đó việc mô hình hóa quá trình lan truyền thông tin trên mạng là nền tảng trong tiếp cận của họ. Các mô hình lan truyền thông tin, hay còn gọi là mô hình khuếch tán thông tin được các nhà khoa học đưa ra đã mô hình hóa toán học và mô tả một cách tương đối chính xác một mạng lưới và quá trình lan truyền thông tin, lan truyền dịch bệnh ở trên mạng lưới đó.

Thúc đẩy bởi những hiện tượng trên và yêu cầu bức thiết của việc giải quyết và ngăn chặn những tác hại do virus và tin đồn trên mạng xã hội mang lại. Nhóm tác giả đã mạnh dạn nghiên cứu đề tài nhằm mục đích tìm ra giải pháp hợp lý để giải quyết những vấn đề nêu trên. Đặc biệt là trong bối cảnh ở Việt Nam ngày càng mở rộng hợp tác quan hệ với các nước trên thế giới, đi kèm theo nhiều lợi ích đó là đối mặt với nhiều nguy cơ thách thức về An ninh quốc gia. Đặc biệt là các thế lực thù địch lợi dụng MXH để thực hiện những âm mưu, hoạt động nhằm xuyên tạc, chống phá đường lối, chính sách của Đảng, pháp luật của Nhà nước. Nghiêm trọng hơn, trong thời gian gần đây, chúng thực hiện những thủ đoạn cực kì tinh vi: Thuê người làm dữ liệu giả (ảnh, video clip, audio hiện trường) để đưa lên mạng xã hội, các diễn đàn, xuyên tạc sự thật, cung cấp thông tin sai lệch về các cá nhân, các cán bộ, Đảng viên; dùng kỹ thuật và công nghệ để chỉnh sửa dữ liệu cũ, chỉnh sửa hình ảnh, cắt ghép, tạo bằng chứng và thông tin giả; tự bịa ra các bài phỏng vấn nhân vật, sự kiện, bịa đặt các trang hồ sơ liên quan đến các nhân vật nổi tiếng, các nhà lãnh đạo, thân nhân của họ và kích thích trí tò mò của công chúng bằng “thông tin lề trái, thông tin bí mật”, những việc làm này gây hậu quả vô cùng nghiêm trọng đối với tình hình An ninh quốc gia của Việt Nam.

\tocless\section{Tình hình nghiên cứu liên quan đến đề tài}
Ngăn chặn sự phát tán của thông tin sai lệch, tin đồn hiện nay là một chủ đề nóng được nhiều nhà khoa học quan tâm. Rất nhiều những nghiên cứu và bằng sáng chế liên quan vấn đề này có tính ứng dụng cao được áp dụng trên các mạng xã hội, ứng dụng trong giáo dục, y tế sức khỏe, ứng dụng cho chính phủ cũng như các vấn đề bảo vệ An ninh Quốc gia. 

Không chỉ ở Việt Nam, đối với các nước trên thế giới, sự phát tán của thông tin sai lệch hiện nay đã trở thành vấn nạn nghiêm trọng. Một số nước đã thành lập các trung tâm chống tin giả sử dụng các biện pháp, giải pháp xã hội cũng như kỹ thuật để ngăn chặn, hạn chế sự phát tán của thông tin sai lệch tới công chúng.

Nguồn phát tán thông tin sai lệch có thể được phát hiện thông qua khảo sát người dùng hoặc các phương pháp khai phá dữ liệu. Ngoài ra, tin đồn còn có thể được phát hiện thông qua việc sử dụng các đặc trưng như: thời gian, cấu trúc và ngữ nghĩa \cite{nguyen9}. Về phương diện mô hình hóa toán học, để phát hiện được nguồn phát tán thông tin sai lệch, Zhang và cộng sự đề xuất vấn đề Time Con-strained Misinformation Detection (TCMD) tìm kiếm tập đỉnh nhỏ nhất sao cho phát hiện được thông tin sai lệch lớn hơn ngưỡng t. Christakis \cite{chris} và cộng sự đề xuất một phương thức phát hiện sự lan truyền của “ổ dịch” bằng cách theo dõi người dùng sử dụng phương pháp lựa chọn ngẫu nhiên.

Nguyen và cộng sự \cite{nguyen9} đề xuất chiến lược cho việc “khử nhiễm” thông tin sai lệch trên mạng xã hội bằng cách dùng thông tin tốt, chính thống để “khử nhiễm” thông tin xấu. Tuy nhiên, khi người dùng đã tin theo thông tin sai lệch sẽ là khó để thay đổi quan điểm của họ và chúng ta cần biết nội dung của thông tin xấu để đưa ra những thông tin chính thống phù hợp.

Gần đây, các nhà khoa học đã đề xuất một chiến lược chung trong việc ngăn chặn sự pháp tán của tin đồn, thông tin sai lệch, đó là ngăn chặn những tài khoản và liên kết có vai trò quan trọng trong quá trình lan truyền, có một số phương pháp heuristic hiệu quả đã được đề xuất. Khalil \cite{khalil} và cộng sự đề xuất bài toán xóa bỏ tập cạnh để giảm thiểu ảnh hưởng của tập nguồn, họ cũng đề xuất một thuật toán giải quyết có tỷ lệ xấp xỉ $1 - 1/e - \varepsilon$. Ceren Budak \cite{budak} nghiên cứu bài toán hạn chế sự lây lan của một chiến dịch lan truyền thông tin “xấu”  trong mạng xã hội, dựa trên các mô hình lan truyền và tính cạnh tranh của thông tin trên mạng xã hội trực tuyến. 

Canh.V Pham \cite{cvpham} và các cộng sự nghiên cứu việc chặn thông tin sai lệch trong một thời gian và ngân sách cho trước. Cui \cite{cui} và cộng sự tập trung vào việc nghiên cứu lựa chọn những điểm đặt các cảm biến sao cho phát hiện sự bùng nổ thông tin với xác suất cao nhất. Zhang đề xuất vấn đề t - Monitors Placement  ngăn ngừa việc làm truyền thông tin sai lệch với ngưỡng đảm bảo t $\in$ [0,1] đề xuất chiến lược tiêm vaccin cho k đỉnh còn lại sao cho số đỉnh bị lây lan là nhỏ nhất trong mô hình IC. Các tác giả cho thấy rằng vấn đề tối ưu hóa này là NP-khó ngay cả việc tính toán hàm mục tiêu. Ngoài ra, có rất nhiều nghiên cứu liên quan đến vấn đề này trong thời gian gần đây.

Tuy nhiên, hầu hết các nghiên cứu trên đều được thực hiện trên dữ liệu có sẵn đã được công bố và chúng bị hạn chế theo những khía cạnh sau:
\begin {itemize}
	\item Việc tính toán hàm mục tiêu là NP-Khó trên các mô hình phát tán thông tin cổ điển dẫn đến việc khó thực hiện giả pháp trên các mạng xã hội thực.
	
	\item Yếu tố thời gian chưa được tập trung xem xét trong ngăn chặn thông tin sai lệch. Đây là điều rất quan trọng vì thông tin chỉ phát tán nhanh trong một khoảng thời gian nhất định, nếu thông tin sai lệch phát tán trong thời gian càng dài thì càng gây nhiều thiệt hại và ảnh hưởng tới công chúng. Hơn nữa, việc ngăn chặn thông tin sai lệch với nguồn lực cho trước có một sự hạn chế là ta không biết ngân sách bao nhiêu để ngăn chặn sự bùng phát của thông tin sai lệch.
	
	\item Các nghiên cứu mới chỉ dừng lại ở mô hình và thuật toán mà chưa kết hợp với việc thu thập dữ liệu hiện thực, đặc biệt chưa áp dụng ở các mạng xã hội ở Việt Nam.
\end {itemize}

Trong đề tài này, nhóm tác giả tập trung khắc phục các hạn chế trên của những nghiên cứu đã có. Đặc biệt, nhóm tác giả đã thu thập dữ liệu của các đối tượng phát tán thông tin sai lệch ở Việt Nam và đưa ra giải pháp ngăn chặn thông tin sai lệch trên các dữ liệu được thu thập đó. Qua đó có thể xây dựng một hệ thống ngăn chặn thông tin sai lệch trên dữ liệu thu thập được bằng việc đưa ra danh sách người dùng mà việc ngăn chặn họ tin theo, nghe theo, phát tán thông tin sai lệch sẽ hạn chế được tối đa thông tin sai lệch.

\tocless\section {Mục tiêu, nhiệm vụ nghiên cứu}
\begin {itemize}
	\item {\bfseries Mục tiêu nghiên cứu:} Đề tài thực hiện để đạt được các mục tiêu sau:
		\begin {enumerate} [+]
			\item Tìm hiểu, làm rõ tác hại của thông tin sai lệch trên các mạng xã hội.
			\item Tìm hiểu cơ chế lây lan tin đồn và virus trên mạng xã hội. Qua đó đề xuất mô hình lan truyền thông tin phù hợp có yếu tố thời gian.
			\item Thiết lập bài toán hạn chế tin đồn và virus trên mạng xã hội trên mô hình lan truyền thông tin mới.
			\item Đưa ra giải pháp, thuật toán hiệu quả để ngăn chặn sự lây lan tin đồn và virus trên mạng xã hội.
			\item Thu thập dữ liệu về người dùng trên mạng xã hội thực, đặc biệt là dữ liệu các đối tượng phát tán thông tin. Đưa ra kết quả thực nghiệm và đánh giá. 
		\end {enumerate}
	\item {\bfseries Nhiệm vụ nghiên cứu: }Để đạt được mục tiêu nghiên cứu, nhóm tác giả giải quyết lần lượt các nhiệm vụ sau:
		\begin {enumerate} [+]
			\item Nghiên cứu mạng xã hội, cách thức lan truyền thông tin trên mạng xã hội, mô hình hóa toán học của mạng xã hội trong các nghiên cứu từ trước tới nay trên các hội nghị khoa học, tạp chí khoa học uy tín qua các tài liệu, bài báo cáo.
			\item Nghiên cứu các thuật toán liên quan đã có giải quyết bài toán ngăn chặn sự lan truyền thông tin sai lệch trên mạng xã hội trực tuyến, cập nhật các nghiên cứu trong thời gian gần đây.
			\item Đưa ra thuật toán hiệu quả cho bài toán ngăn chặn sự lây lan của tin đồn và virus trên mạng xã hội, so sánh với các thuật toán mới nhất. Nghiên cứu mô hình hóa mạng xã hội và quá trình lan truyền thông tin trên mạng xã hội.
			\item Nghiên cứu giải pháp lấy dữ liệu trên các mạng xã hội, trong phạm vi của đề tài là mạng xã hội Facebook.
			\item Đề xuất giải pháp hạn chế thông tin sai lệch trên mạng xã hội và thực nghiệm kiểm tra giải pháp.
			\item Thực nghiệm giải pháp trên dữ liệu thực thu thập từ mạng xã hội Facebook. Đưa ra danh sách người dùng có vai trò nhất trong việc phát tán thông tin sai lệnh từ một người cho trước. Từ đó đưa ra những giải pháp thích hợp với những người dùng này.
		\end {enumerate}
\end {itemize}

\tocless\section {Đối tượng và phạm vi nghiên cứu của đề tài}
\begin {itemize}
	\item {\bfseries Đối tượng nghiên cứu:}
		\begin{enumerate} [+]
			\item Các mạng xã hội trên thế giới đặc biệt là các mạng xã hội phổ biến ở Việt Nam được đa số người dùng sử dụng.
			\item Hành vi phát tán thông tin trên mạng xã hội.
			\item Các đối tượng phát tán thông tin sai lệch.
		\end{enumerate}
	\item {\bfseries Phạm vi nghiên cứu}
		\begin {enumerate} [+]
			\item Thời gian: Các mạng xã hội từ năm 2001 tới nay.
			\item Không gian: Mạng Internet.
		\end {enumerate}
\end {itemize}

\tocless\section {Phương pháp nghiên cứu}
Trên cơ sở những điều kiện và đặc điểm nêu trên, để đạt được mục tiêu nghiên cứu của đề tài, nhóm tác giả sử dụng tổng hợp các cách tiếp cận và phương pháp nghiên cứu sau:
\begin {itemize}
	\item {\bfseries Phương pháp nghiên cứu tài liệu:}
		\begin {enumerate} [+]
			\item Phương pháp phân tích và tổng hợp lý thuyết: Nhóm tác giả đã tham khảo các bài báo, tạp chí, hội nghị có uy tín trong nước và thế giới về những nghiên cứu có liên quan đến đề tài, từ đó phân tích đánh giá, có cái nhìn gần hơn, ví dụ như :
				\begin {enumerate} [$\bullet$]
					\item Science direct: http://www.sciencedirect.com/.
					\item IEEE Xplore Digital Library: http://ieeexplore.ieee.org.
					\item The ACM Digital Library: http://dl.acm.org/.
					\item Google Scholar: https://scholar.google.com.
				\end {enumerate}
			\item Phương pháp phân loại và hệ thống hóa lý thuyết: Từ những lý thuyết, tài liệu đã có, nhóm tác giả phân loại, hệ thống hóa, sắp xếp một cách khoa học, dễ hiểu. Đồng thời, có những nhận định, đánh giá về những lý thuyết, công trình nghiên cứu trước đây, qua đó, nhóm tác giả có những đề xuất tìm ra phương pháp giải quyết tối ưu hơn cho đề tài.
		\end {enumerate}
	\item {\bfseries Phương pháp nghiên cứu thực tiễn}
		\begin {enumerate} [+]
			\item Phương pháp hỏi chuyên gia: Thu thập thông tin tài liệu, ý kiến của chuyên gia, đánh giá vấn đề, hỏi ý kiến từ các chuyên gia hàng đầu về lĩnh vực công nghệ thông tin và an toàn thông tin để có cái nhìn tổng quát hơn về vấn đề đang nghiên cứu.
			\item Từ thực tiễn đề xuất mô hình phù hợp với thực tế hơn.
			\item Phương pháp mô hình hóa: Từ những mô hình mạng xã hội, nhóm tác giả mô hình hóa thành các bài toán phù hợp, từ đó sẽ giải quyết những yêu cầu của bài toán bằng các công cụ của toán học, khoa học máy tính,...
			\item Phương pháp thống kê: Tổng hợp, thống kê những số liệu thu được, mô hình hóa bằng các biểu đồ để có cái nhìn tổng quát hơn, từ đó dễ dàng đánh giá, nhận định vấn đề, đánh giá thuật toán và mô hình đề xuất.
		\end {enumerate}
\end {itemize}

\tocless\section {Ý nghĩa khoa học, ý nghĩa thực tiễn của đề tài}
\begin {itemize}
	\item {\bfseries Ý nghĩa khoa học}: 
		\begin {enumerate} [+]
			\item Cung cấp cái nhìn tổng quan về thông tin sai lệch và thực trạng lan truyền thông tin sai lệch hiện nay.
			\item Đưa ra những lý thuyết mới có ý nghĩa, những giải pháp có tính khả thi trong nghiên cứu cơ bản cũng như thực tiễn hạn chế thông tin sai lệch trên mạng xã hội.
			\item Đưa ra được mô hình mới và thuật toán mới khắc phục được những hạn chế của các mô hình và thuật toán đã công bố.
		\end {enumerate}
		Qua đó cho thấy, một phần công trình nghiên cứu của nhóm được các đồng nghiệp trên thế giới đánh giá cao và có giá trị về mặt học thuật cũng như giá trị trong thực tiễn. Nhóm tác giả đã vận dụng những nghiên cứu quan trọng này trong đề tài, đề đưa ra giải pháp ngăn chặn thông tin sai lệch có hiệu quả.
	\item {\bfseries Ý nghĩa thực tiễn}:
		\begin {enumerate} [+]
			\item Giúp người dùng có cái nhìn rõ hơn về những tác hại của mạng Internet và đặc biệt là thực trạng về thông tin sai lệch.
			\item Đánh giá thực trạng lan truyền, quy luật lan truyền của thông tin sai lệch trên mạng xã hội. Đề xuất một giải pháp hạn chế thông tin sai lệch trên mạng xã hội.
			\item Là nền tảng để xây dựng một hệ thống để chặn thông tin sai lệch và có ý nghĩa trong thực tiễn. Đặc biệt trong bối cảnh hiện nay, sự phát tán rộng rãi của thông tin sai lệch là vấn nạn đối với các quốc gia. Một số nước phát triển đã xây dựng trung tâm phòng chống tin giả.
			\item Qua thu thập dữ liệu người phát tán thông tin ta góp phần vào công tác Công an trong điều tra vụ án.
			\item Từ việc ngăn chặn thông tin sai lệch, ta hạn chế được rất nhiều những hậu quả nghiêm trọng do thông tin sai lệch gây ra.
		\end {enumerate}
\end {itemize}

\tocless\section {Cấu trúc của đề tài - CẦN CHỈNH SỬA}
Ngoài phần mở đầu, kết luận, danh mục tài liệu tham khảo và phụ lục, đề tài được cấu trúc gồm 4 chương:

{\bfseries Chương 1: Giới thiệu về mạng xã hội và tác hại của thông tin sai lệch}

Chương này giới thiệu tổng quan về mạng xã hội bao gồm: Định nghĩa mạng xã hội, những đặc trưng cơ bản, lợi ích và tác hại của mạng xã hội. Đặc biệt, trong chương này trình bày thực trạng lan truyền thông tin sai lệch trên mạng xã hội đối với thế giới và Việt Nam, từ đó đặt ra vấn đề cấp thiết trong việc tìm hiểu thông tin sai lệch, cơ chế lan truyền và giải pháp hạn chế tác hại của nó.

{\bfseries Chương 2: Thông tin sai lệch và cơ chế lan truyền thông tin sai lệch}

Chương này nhóm tác giả trình bày định nghĩa thông tin sai lệch, những nguy cơ và hậu quả do thông tin sai lệch gây ra đối với các cá nhân, tổ chức. Đồng thời, phân tích cơ chế lan truyền thông tin và những đặc tính của hai mô hình lan truyền thông tin đang được sử dụng rộng rãi bao gồm: Mô hình tầng độc lập và mô hình ngưỡng tuyến tính. Ngoài ra, ở Chương 2 tổng quan một số hướng nghiên cứu liên quan đến bài toán hạn chế lan truyền thông tin sai lệch trên mạng xã hội trực tuyến.

{\bfseries Chương 3: Giải pháp hạn chế thiệt hại do thông tin sai lệch gây ra trên mạng xã hội trực tuyến}

Từ thực trạng đã nêu trong Chương 2 và xuất phát từ những công trình nghiên cứu liên quan trước đó, nhóm tác giả đã phát biểu tổng quát bài toán Cực tiểu hóa thiệt hại do thông tin sai lệch gây ra trên mạng xã hội trực tuyến, với dữ liệu thu thập trực tiếp từ mạng xã hội Facebook xung quanh các đối tượng liên quan đến các vấn đề xâm phạm An ninh Quốc gia, gây rối trật tự an toàn xã hội. Nghiên cứu, chứng minh bài toán thuộc lớp \#P – khó, đồng thời đề xuất các giải pháp hiệu quả để giải quyết bài toán này.