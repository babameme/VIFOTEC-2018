\chapter{GIỚI THIỆU VỀ MẠNG XÃ HỘI VÀ TÁC HẠI CỦA THÔNG TIN SAI LỆCH }

\section{Giới thiệu chung về mạng xã hội}
Mạng xã hội (MXH), hay còn gọi là mạng xã hội ảo (Social Network) là dịch vụ nối kết các thành viên cùng sở thích trên Internet với nhiều mục đích khác nhau không phân biệt không gian và thời gian. Những người tham gia vào MXH còn được gọi là “cư dân mạng”. 

MXH bao gồm rất nhiều các dịch vụ mang các tính năng riêng biệt: chat, mail, blog, video, chia sẻ thông tin giúp kết nối người dùng dựa vào các mối quan hệ đặc trưng như: Quan hệ họ hàng, sở thích, ý tưởng. Chính vì thế, mạng xã hội dễ dàng giúp con người có thể tìm kiếm kết nối với các mối quan hệ dựa trên các nhóm, trường, cơ quan, dựa trên các thông tin cá nhân, địa chỉ. 

MXH bao gồm 2 đặc điểm cơ bản. Đặc điểm thứ nhất là có sự tham gia trực tuyến của các cá nhân hay các chủ thể. Đặc điểm thứ hai là mạng xã hội sẽ có các trang web mở, người dùng tự xây dựng nội dung trong đó và các thành viên trong nhóm đấy sẽ biết được các thông tin mà người dùng viết.

Hiện nay thế giới có rất nhiều mạng xã hội khác nhau, thị trường Bắc Mỹ và Tây Âu nổi tiếng với MySpace và Facebook; Nam Mỹ với Orkut và Hi5; Friendster tại Châu Á và các đảo quốc Thái Bình Dương. Mạng xã hội khác gặt hái được thành công đáng kể theo vùng miền như Bebo tại Anh Quốc, CyWorld tại Hàn Quốc, Mixi tại Nhật Bản. Tại Việt Nam, theo một nghiên cứu mới đây của DoubleClick Ad Planner, các trang MXH có lượng truy cập gần 16 triệu lượt/tháng trong đó có ba mạng xã hội lớn nhất tại Việt Nam: Facebook, Zing Me và Yume.

Mục tiêu là tạo ra một hệ thống trên nền Internet cho phép người dùng giao lưu và chia sẻ thông tin một cách có hiệu quả, vượt ra ngoài những giới hạn về địa lý và thời gian. Xây dựng lên một mẫu định danh trực tuyến nhằm phục vụ những yêu cầu công cộng chung và những giá trị của cộng đồng. Qua đó, nhằm nâng cao vai trò của mỗi công dân trong việc tạo lập quan hệ và tự tổ chức xoay quanh những mối quan tâm chung trong những cộng đồng thúc đẩy sự liên kết các tổ chức xã hội.
	\subsection{Đặc điểm của xã hôi}
	MXH trên Internet bao gồm các đặc điểm nổi bật: Tính liên kết cộng đồng, tính tương tác, khả năng truyền tải và lưu trữ lượng thông tin khổng lồ.
		\subsubsection{Tính liên kết cộng đồng}
		Đây là đặc điểm nổi bật của mạng xã hội ảo cho phép mở rộng phạm vi kết nối giữa người với người trong không gian đa dạng. Người sử dụng có thể liên kết với nhau, việc liên kết này tạo ra một cộng đồng mạng với số lượng thành viên lớn.
		\subsubsection{Tính đa phương tiện}
		MXH có rất nhiều tiện ích nhờ sự kết hợp giữa các yếu tố chữ viết, âm thanh, hình ảnh, Sau khi đăng kí tài khoản người sử dụng có thể tạo ra một không gian riêng cho bản thân. Nhờ các tiện ích đa phương tiện mà người sử dụng có thể chia sẻ thông tin, hình ảnh, video. Đặc điểm này được phản ánh trong cấu trúc phân lớp ứng dụng của MXH.
		\subsubsection{Tính tương tác}
		Tính tương tác được thể hiện không chỉ ở thông tin được di truyền và sau đó nhận được phản hồi từ phía người nhận mà còn phụ thuộc vào cách người dùng sử dụng các ứng dụng
		\subsubsection{Khả năng truyền tải và lưu trữ lượng thông tin khổng lồ}
		Tất cả các MXH đều có những ứng dụng tương tự nhau như đăng trạng thái, nhạc, video clip, viết bài nhưng được phân bố với dung lượng khác nhau. Các trang MXH lưu trữ thông tin và nhóm sắp xếp chúng theo một thứ tự thời gian, nhờ đó người sử dụng có thể truy cập và tìm kiếm thông tin.
	\subsection{Lợi ích của mạng xã hội}
	Kể từ khi có kết nối mạng trên toàn cầu (Internet) và nhất là sau khi điện thoại thông minh hay máy tính bảng được chế tạo, việc sử dụng các MXH như Facebook, Instagram, Viber, Zalo, Skype, Whatsapp, Youtube, Linked, Twitter đã không còn xa lạ với hầu hết người dùng, kể cả trẻ em, thanh thiếu niên và người lớn tuổi. Thế giới ngày càng phát triển, mạng xã hội càng giúp con người xích lại gần nhau hơn, đem lại những tính năng và lợi ích vô cùng tuyệt vời.
		\subsubsection{Kết nối bạn bè, gia đình, cộng đồng}
		Ngày nay, con người ngày càng có ít thời gian cho bản thân và mở rộng các mối quan hệ. Nhờ có mạng xã hội, thông qua văn bản, video, hình ảnh con người có thể kết nối với nhau rất thuận tiện. Ngoài ra họ cũng có thể mở rộng các mối quan hệ khác về mọi lĩnh vực mà người dùng quan tâm. Người dùng có thể kết bạn với nhiều nhóm người với những sở thích, sở trường khác nhau. Hầu hết các mạng xã hội đều yêu cầu người dùng để đưa ra một số thông tin nhất định thường bao gồm: Độ tuổi, giới tính, địa điểm, quan điểm, sở thích... Tuy nhiên, những thông tin rất cá nhân thường không được khuyến khích vì lý do an toàn. Điều này cho phép người dùng khác tìm kiếm theo một số loại tiêu chuẩn phù hợp đối với mình và duy trì một mức độ ẩn danh tương tự như hầu hết các dịch vụ hẹn hò trực tuyến.
		\subsubsection{Cập nhật tin tức, kiến thức, xu thế}
		Tin tức sẽ được cập nhật theo từng giây, trong học tập, nghiên cứu thì đây cũng là một kênh tin tức bổ ích. Theo báo cáo của Hội Liên hiệp giáo dục Mỹ (The National School Boards Association), 60\% sinh viên sử dụng mạng xã hội nói chuyện về chủ đề giáo dục trực tuyến, và hơn 50\% nói chuyện cụ thể về việc học ở trường. Một số mạng xã hội khác như: TermWiki, Learn Central và các trang web khác được xây dựng để thúc đẩy mối các quan hệ trong giáo dục bao gồm các Blog giáo dục, ePortfolios cũng như thông tin liên lạc như chat, bài thảo luận, và các diễn đàn học tập.
		\subsubsection{Cải thiện chất lượng và tốc độ của báo chí và dịch vụ công}
		Do tính năng cập nhật và lan rộng nhanh của MXH mà các cơ quan báo chí và thông tin đại chúng đang tích cực đăng tải cùng một lúc trên báo giấy, trên báo điện tử và trang mạng của mình để theo kịp xu thế của thời đại và giữ số lượng độc giả của mình. Các cơ quan pháp luật hay dịch vụ công cũng đang dần “lên sóng” MXH để cập nhật những tin tức và quy định mới của mình hoặc lắng nghe ý kiến phê bình góp ý của người dân nhằm giúp giảm thiểu sự quan liêu, phức tạp hay sai sót trong dịch vụ công, để tiến tới một bộ máy hành chính công thông minh và giản tiện hơn.
		\subsubsection{Cải thiện kĩ năng sống, kiến thức}
		Hiện nay trên các mạng xã hội xuất hiện ngày càng nhiều các trang dạy ngoại ngữ, nấu ăn, sửa chữa, giao tiếp, tâm lý, thể thao để xem tham khảo, tự học mà không cần đến lớp hay đóng lệ phí. Chính nhờ tham gia các cộng đồng mạng này, chúng ta đang ngày càng trở nên hoàn thiện hơn với những kỹ năng cơ bản cần thiết trong cuộc sống hiện đại như sử dụng ngoại ngữ, cách giao tiếp văn minh hay có một thể hình khỏe đẹp.
		\subsubsection{Kinh doanh, quảng cáo miễn phí}
		Rất nhiều công ty, nhà quảng cáo đã sử dụng mạng xã hội để bán hàng, quảng cáo cho sản phẩm của mình. Ở các trang mạng xã hội như Facebook, Instagram,... có rất nhiều người trẻ khởi nghiệp bằng cách bán hàng online. Mạng xã hội kết nối con người với chi phí thấp, có những chức năng phù hợp, do đó đem lại hiệu quả cao, kể cả việc tư vấn cho người dùng cũng trở nên nhanh chóng, dễ dàng. Việc kinh doanh, quảng cáo sẽ trở nên phổ biến hơn do chức năng “chia sẻ” từ đó sẽ có nhiều hơn những người dùng đọc được thông tin về sản phẩm.
		\subsubsection{TIết kiệm kinh phí, thời gian, sức lao động}
		Nhờ MXH mà công ty, tổ chức hay hộ gia đình đã tiết kiệm được chi phí giấy, mực in, nhân công, phí điện thoại, tin nhắn. Nếu như ngày trước chúng ta phải di chuyển rất xa, rất mất thời gian, công sức, tiền bạc thì mới có thể truyền được thông tin, thì nay nó đã quá đơn giản, chỉ cần một cú nhấp chuột, tất cả thông tin được chuyển đi nhanh chóng và dễ dàng. Một số mạng xã hội nhằm mục đích khuyến khích lối sống lành mạnh đối với người dùng. Ví dụ như: Mạng xã hội SparkPeople cung cấp cho cộng đồng các công cụ trợ đồng đẳng trong việc giảm cân, Fitocracy tập trung vào hướng dẫn người dùng trong tập thể dục hoặc cho phép người dùng chia sẻ tập luyện của mình và nhận xét về những người dùng khác.
		\subsubsection{Tác động chính trị, xã hội}
		Nếu được sử dụng đúng cách, mạng xã hội có vai trò quan trọng đối với các chính trị gia, giúp họ được nhiều người dân biết đến hơn, giúp họ tự xây dựng hình ảnh trong mắt công chúng. Từ đó, có thể thúc đẩy sự nghiệp của họ. Ví dụ điển hình cho việc sử dụng mạng xã hội thành công là Tổng thống Mỹ Donald Trump. Gần đây, ông Donald Trump đã khẳng định: “Tôi nghĩ có thể tôi đã không ngồi ở vị trí này nếu không có Twitter. Twitter là một thứ tuyệt vời đối với tôi, vì tôi có thể truyền tải suy nghĩ của mình tới công chúng. Tôi có thể sẽ không ở đây nói chuyện với tư cách tổng thống nếu không thể đưa ra những phát biểu chân thật”.
		
		Dù vậy, MXH cũng là “con dao hai lưỡi” nếu ta sử dụng không đúng mục đích. Sử dụng MXH quá nhiều sẽ dẫn đến xao nhãng trong học tập, mất đi thời gian vận động, thể dục thể thao. Do nguồn thông tin trên mạng không có ai giám sát, kiểm duyệt nên còn tràn lan rất nhiều thông tin sai lệch, văn hóa phẩm đồi trụy, trong khi giới trẻ còn chưa đủ nhận thức để sàng lọc thông tin, dễ dẫn đến nhận thức lệch lạc, kéo theo đó là hành động sai lầm như: giết người, nghiện hút, mại dâm...Bên cạnh đó sử dụng MXH quá nhiều còn dẫn đến mất khả năng tương tác giữa mọi người, có nguy cơ mắc bệnh trầm cảm, xao nhãng những mục tiêu thật của cuộc sống. Việc bảo mật thông tin chưa thực sự tốt, nghiêm trọng nhất là nguy cơ lây lan thông tin, đặc biệt là những thông tin sai lệch trên MXH, sẽ dẫn đến những hậu quả vô cùng nghiêm trọng.
		
\section{Sự lây lan và tác hại của thông tin sai lệch trên mạng xã hội}
Trong phần này, nhóm tác giả trình bày định nghĩa thông tin sai lệch, các tính chất của thông tin sai lệch, thực trạng phát tán của thông tin sai lệch trên thế giới và thực trạng ở Việt Nam, đi kèm theo đó là các vụ việc thực tế đã và đang xảy ra trong thời gian gần đây.
	\subsection{Định nghĩa thông tin sai lệch và tính chấn}
		\subsubsection{Định nghĩa thông tin sai lệch}
		Trong thực tế trên MXH luôn tồn tại những thông tin lệch lạc, không lành mạnh gây ra ảnh hưởng tiêu cực đến người dùng bên cạnh những giá trị tích cực mà chúng mang lại. Trong thực tế, bên cạnh các thông tin bổ ích, có giá trị đối với xã hội thì còn vô số thông tin, hình ảnh có nội dung xấu độc. Tại khoản 1, điều 5 Nghị định 72/2013/NĐ-CP ngày 15/7/2013 của Chính phủ đã có quy định chi tiết về việc quản lý, cung cấp, sử dụng dịch vụ Internet và thông tin trên mạng \cite{quidinh}. Trong đó có nhiều hành vi bị nghiêm cấm như lợi dụng việc cung cấp, sử dụng dịch vụ Internet và thông tin trên mạng nhằm mục đích chống lại Nhà nước Cộng hòa xã hội chủ nghĩa Việt Nam; gây phương hại đến an ninh quốc gia, trật tự an toàn xã hội; phá hoại khối đại đoàn kết dân tộc; tuyên truyền chiến tranh, khủng bố; gây hận thù, mâu thuẫn giữa các dân tộc, sắc tộc, tôn giáo (điểm a). Tuyên truyền, kích động bạo lực, dâm ô, đồi trụy, tội ác, tệ nạn xã hội, mê tín dị đoan, phá hoại thuần phong, mỹ tục của dân tộc (điểm b). Tiết lộ bí mật nhà nước, bí mật quân sự, an ninh, kinh tế, đối ngoại và những bí mật khác do pháp luật quy định (điểm c). Đưa thông tin xuyên tạc, vu khống, xúc phạm uy tín của tổ chức, danh dự và nhân phẩm của cá nhân (điểm d). Quảng cáo, tuyên truyền, mua bán hàng hóa, dịch vụ bị cấm; truyền bá tác phẩm báo chí, văn học, nghệ thuật, xuất bản phẩm bị cấm (điểm đ). Giả mạo tổ chức, cá nhân và phát tán thông tin giả mạo, thông tin sai sự thật xâm hại đến quyền và lợi ích hợp pháp của tổ chức, cá nhân (điểm e).

		Theo Karlova và Fisher, 2013 \cite{karlova1}, thông tin sai lệch được hiểu là những thông tin giả mạo, không chính xác. Dựa trên mục đích của người lan truyền, thông tin sai lệch được phân thành hai loại:
		\begin {itemize}
			\item {\itshape Thông tin sai lệch lan truyền vô ý}: Thông tin sai lệch được tạo ra và lan truyền một cách vô ý, không có chủ đích. Mọi người có xu hướng giúp lan truyền những thông tin như vậy do niềm tin với bạn bè, người thân và ảnh hưởng của họ trên MXH.
		
			\item {\itshape Thông tin sai lệch lan truyền cố ý}: Đó là những tin đồn, tin tức giả mạo, sai lệch được tạo ra và lan truyền một cách cố ý bởi người dùng với mục đích, động cơ không trong sáng.
		\end {itemize}
		Như vậy, có thể thấy rằng, mặc dù có những định nghĩa khác nhau về thông tin sai lệch tuy nhiên những khái niệm có những điểm tương đồng giống nhau. Đó đều là những thông tin không đảm bảo tính chính xác hoặc thông tin giả mạo, xuyên tạc vấn đề, xuyên tạc nội dung v.v… gây ảnh hưởng xấu đến cá nhân và tổ chức, đồng thời mỗi quốc gia có những quy định riêng về những hành vi bị cấm khi đưa thông tin lên mạng và đề được cụ thể hóa trong văn bản pháp luật.
		
		Xuất phát từ những thực tế nêu trên, nhóm tác giả nhận thấy việc ngăn chặn, hạn chế kịp thời sự lan truyền của thông tin sai lệch trên MXH là vô cùng cấp thiết nhằm giảm thiểu tối đa những thiệt hại do chúng gây ra đối với người dùng, góp phần làm trong sạch môi trường mạng, nâng cao sự tin tưởng của người dùng đối với với những thông tin trên MXH. 
		\subsubsection{Tính chất}
		\begin {itemize}
			\item {\itshape Thông tin xuyên tạc có thể có tốc độ lây lan nhanh, dễ dàng: }Trong thời đại truyền thông 4.0, MXH trở thành một môi trường thích hợp để thông tin xuyên tạc lây lan nhanh chóng bởi những đặc điểm của nó. MXH là những website mở, nội dung được xây dựng hoàn toàn bởi các thành viên tham gia, hơn thế nữa, MXH có sự tham gia trực tiếp của nhiều cá nhân hay các chủ thể. Qua đó, các thông tin được dư luận và các cá nhân cung cấp có thể mập mờ, không chính xác là cơ sở cho việc phát sinh thông tin xuyên tạc, sai lệch. Hơn thế nữa, người đọc thường có xu hướng quan tâm đến các tin đồn nhiều hơn tin chính thống vì chúng có thể gây nên nhiều liên tưởng tò mò, hấp dẫn. MXH Facebook cho phép người dung đăng tin lên cùng với các chức năng bình luận, like, share. Chính vì vậy, thông tin sai lệch thông qua những đặc điểm này có thể lan truyền với tốc độ chóng mặt và ngày càng nhiều diễn biến phức tạp.
		
			\item {\itshape Khả năng lan truyền rộng, khó kiểm soát được tính xác minh của thông tin: }Ban đầu thông tin sai lệch được đăng tải lên bởi một cá nhân hoặc tổ chức nhưng chưa được xác mình hoặc ghi nhận bởi các cơ quan thẩm quyền, thông tin này thường được xuất phát từ các nhóm nhỏ và thường mang nội dung về các vấn đề chính trị, về cá nhân tổ chức khác, chúng được thảo luận, chia sẻ qua các tính năng của mạng xã hội đến nhiều người dùng để tạo cơ sở để công chúng tin tưởng hoặc bị ảnh hưởng. Khi thông tin đó đã trở nên phổ biến, có được sự quan tâm rộng rãi của công chúng, trong quá trình truyền từ người này sang người khác, các đối tượng xấu đã bóp méo dần sự thật, thêm thắt gây ra hậu quả vô cùng to lớn.
		
		Thế nhưng, hiện nay các mạng xã hội này vẫn chưa chính xác một cơ chế để kiểm duyệt những thông tin này có phù hợp không, người dùng cũng thiếu các thông tin xác thực để kiểm chứng, thẩm định dẫn đến hiểu nhầm, hiểu sai về bản chất của vấn đề.
		\end {itemize}
	\subsection{Thực trạng sự phát tán của thông tin sai lệch trên mạng xã hội }
	Không chỉ ở Việt Nam mà sự phát tán diện rộng của thông tin trên toàn thế giới đã trở thành một vấn nạn chưa từng có. Mạng xã hội trở thành một trong những ưu tiên hàng đầu của các cơ quan hành chính cũng như tư nhân áp dụng vào nhiều mục đích khác nhau. Sử dụng công cụ MXH là một trong các cách nhanh chóng và dễ dàng để tiếp nhận những ý kiến của công chúng và để cho công chúng cập nhật hoạt động của họ. Tuy nhiên, điều này đi kèm với những nguy cơ quá lạm dụng các trang MXH.
	
	MXH ảnh hưởng rất nhiều đến các hoạt động và các mối quan hệ trong thế giới thực. Trong đó, những tin tức về giải trí được quan tâm nhất. Khi người dùng đọc các tin tức mà họ quan tâm, họ có nhiều khả năng sẽ duy trì thảo luận quanh thông tin đó. Ngoài ra, khi nội dụng thông tin liên quan đến vấn đề chính trị, người dùng có nhiều khả năng đưa ra quan điểm, ý kiến của mình về chính trị.
		\subsubsection{Thực trạng thế giới}
		Với số người sử dụng các mạng xã hội trên toàn cầu vào khoảng 3 tỷ người và không có dấu hiệu dừng lại như hiện nay, thông tin sai lệch có sức ảnh hưởng vô cùng lớn tới tình hình thế giới theo nhiều khía cạnh. Sự lây lan thông tin độc hại ảnh hưởng mạnh đến tình hình kinh tế - chính tri, không những thế các đối tượng xấu sử dụng thông tin sai lệch để lừa đảo, chiếm đoạt ảnh hưởng đến tâm lí cũng như sức khỏe của người dùng.
		 
		Ngày 23 tháng 4 năm 2013 tin tặc giả mạo hãng thông tấn Associated Press tung tin Nhà Trắng bị đánh bom và cựu Tổng thống Obama bị thương nặng bởi một vụ  nổ ở Nhà trắng. Ngay lập tức thông tin này làm thị trường tài chính chứng khoán của Mỹ chao đảo. Các chỉ số chứng khoán gần như sụp đổ bởi thông tin này. Chỉ số Down Jones ngay lập tức sụt giảm đến 143 điểm gây thiệt hại 136,5 tỷ USD cho thị trường. Mặc dù vậy, thị trường chỉ rơi vào khoảng lặng hơn 1 phút trước khi AP thông báo đó là tin giả mạo do tài khoản Twitter của báo này bị tin tặc chiếm quyền điều khiển và đăng tin sai sự thật \cite{APhack}.
		
		Gần đây những thông tin sai lệch trên các MXH còn được cho là có ảnh hưởng không nhỏ tới cuộc bầu cử ở Pháp và ở Mỹ. Trong cuộc bầu cử tổng thống ở Pháp Facebook đã xóa 30.000 tài khoản giả mạo báo cáo tin đồn ở Pháp trước cuộc bầu cử Tổng thống vào năm 2017 \cite{Anhhuong}. Trong cuộc bầu cử ở Mỹ MXH được cho ảnh hưởng không nhỏ tới kết quả bầu cử Tổng thống năm 2016. Nhiều tài khoản giả mạo được tạo ra chia sẻ những thông tin sai lệch về sợ rò rỉ email của bà Hilary Clinton và các đồng sự cấp cao của bà. Các tài khoản này đã phát động một chiến dịch phản đối bà Hilary đây được cho là một trong những nguyên nhân lớn đưa đến sự thất bại của bà.
		\subsubsection{Thực trạng tại Việt Nam}
		MXH trong những năm gần đây ngày càng trở nên thịnh hành tại Việt Nam. Cũng như trên thế giới, MXH (điển hình như Facebook) được nhiều người Việt Nam coi là tin tưởng sử dụng. Chính vì vậy, các đối tượng luôn tìm cách lợi dụng điều này để có thể trục lợi cho bản thân, hay thực hiện những hoạt động chống phá, gây ảnh hưởng nghiêm trọng tới uy tín của cá nhân, tập thể, chính quyền.
		
		Vào tháng 8 năm 2014, trên MXH lan truyền nhanh chóng nội dung tin đồn thất thiệt rằng “dịch Ebola đã bùng phát tại Hà Nội”. Trước đó, nhiều người đã truyền nhau thông tin tại Hà Nội đã có người nhiễm Ebola. Thông tin ngay sau đó đã được lan truyền một cách chóng mặt, gây nên một sự hoang mang lo lắng đến người dân. Bên cạnh đó còn có tin đồn cho rằng, nước láng giềng của Viêt Nam là Campuchia đã xuất hiện ca nhiễm virus Ebola \cite{ebola}.
		
		Tình trạng “ô nhiễm” thông tin xuất phát từ hoạt động phá hoại tư tưởng đồng thời phát tán các tư tưởng chống phá Đảng và nhà nước kích động biểu tình bạo loạn thông qua MXH của các thế lực thù địch, phản động chống đối ở trong và ngoài nước. Theo thống kê của cơ quan An ninh, tính đến nay có hơn 2500 trang web, blog, MXH của các cá nhân, tố chức đang hoạt động, đăng tải các tin, bài viết, bình luận, bài phỏng vấn với mục đích tuyên truyền phá hoại tư tưởng nhằm phá hoại tư tưởng. Trong đó nổi lên một số trang như: danlambao, quanlambao, danluan. Các đối tượng quản trị những trang web này lợi dụng những điểm nóng về chính trị, xã hội và những thiếu sót trong công tác quản lý của ta để đăng tin xuyên tạc, kích động biểu tình, bạo loạn. Điển hình là hai vụ việc sự cố môi trường ở 4 tỉnh miền Trung do Formosa xả thải \cite{formusa} và vụ việc khiếu kiện đất đai ở Đồng Tâm mà đỉnh điểm là vụ bắt giữa 38 chiến sĩ Cảnh sát cơ động. Những thông tin sai sự thật này gây phức tạp thêm tình hình gây khó khăn cho công tác giải quyết của Chính quyền Nhà nước.
		
		MXH nơi để mọi người cùng nhau chia sẻ mọi điều trong cuộc sống và đồng thời cũng là nơi để mọi người cập nhật những tin tức, những hình ảnh mới, kết nối với nhau. Nhưng cũng chính từ đây, các chuyện hư cấu, tin đồn, chuyện bóp méo được đăng tải để “câu like”, gây ảnh hưởng đến cộng đồng, dư luận. Để dễ dàng thu hút sự tham gia từ cư dân mạng, không ít các trang fanpage đã sử dụng các chiêu trò câu “like” bằng cách đăng các mẩu tin tức có nội dung đồi trụy; hoặc các ảnh về những trường hợp đau lòng, lợi dụng lòng thương của cư dân mạng để kêu gọi hành động chia sẻ thông điệp.
		
		Những chiêu thức, trò đùa ác ý với nhiều mục đích khác nhau đã làm cho nhiều người hoang mang và tỏ ra e ngại khi tiếp nhận các thông tin trên MXH. Và không chỉ làm hoang mang dư luận, những tin đồn thất thiệt này đã trực tiếp làm ảnh hưởng đến cuộc sống của những người vô tình trở thành nạn nhân. Chỉ bằng một cú click vào xem, chia sẻ, người dùng có thể rơi vào cái bẫy khiến tài khoản cá nhân bị nguy hiểm, bị đánh cắp thông tin. Trong những đường dẫn chứa thông tin sai lệch này còn có thể kèm virus hoặc những phần mềm gián điệp nhằm lấy cắp thông tin hay chiếm quyển kiểm soát máy.
	\subsection*{Kết luận chương}
	\addcontentsline{toc}{subsection}{Kết luận chương}
	Sự ảnh hưởng rộng lớn của thông tin sai lệch đến với người dùng cũng như các công ty, doanh nghiệp ngày càng tăng mạnh nên việc tổ chức hạn chế sự ảnh hưởng của các loại thông tin này đang trở nên cấp bách hơn bao giờ hết. 
	
	Hậu quả của thông tin sai lệch trên MXH là vô cùng nghiêm trọng về mọi mặt chính trị, kinh tế, xã hội. Bản thân của cá nhân, tổ chức bị tung tin sai sự thật phải gánh chịu hậu quả, phiền toái không đáng có, thậm chí là những thiệt hại nặng nề về kinh tế, danh dự, phẩm chất. Thông tin sai lệch có thể gây ảnh hưởng nặng đến kinh tế của người dân, điển hình là phóng sự của VTV về nghề xuất khẩu lao động điều dưỡng viên tại đức có lương tháng 100 triệu \cite{dieuduong}. Thông tin sai lệch này khiến người dùng bị nhầm lẫn dẫn đến đầu tư tiền vào sai chỗ và gây tâm lý chán nản với người đi xuất khẩu lao động. Nguy hại hơn những thông tin sai lệch về chính trị, đường lối chính sách của Đảng và Nhà nước còn gây mất lòng tin của nhân dân vào bộ máy. Những tin này là “mồi dẫn” để các thế lực thù địch tập hợp, lôi kéo lực lượng trên không gian mạng, tổ chức các cuộc biểu tình, bạo loạn, gây mất an ninh trật tự. Ở khía cạnh khác đường link chia sẻ các loại tin sốc, bịa đặt được các hacker sử dụng để phát tán mã độc là bàn đạp cho các cuộc tấn công APT (Tấn công mạng sử dụng công nghệ cao), lừa đảo trên không gian mạng.
	
	Trước những thực trạng to lớn hiện nay, nhóm đã mạnh dạn nghiên cứu, thực nghiệm để có thể đưa ra một giải pháp tối ưu để có thể ngăn chặn tác hại của lan truyền thông tin sai lệch trên mạng xã hội được trình bày ở chương sau.
			