\chapter{CƠ CHẾ LAN TRUYỀN THÔNG TIN SAI LỆCH}
\label{chap:2}
Trong thực tế trên MXH luôn tồn tại những thông tin lệch lạc, không lành mạnh gây ra ảnh hưởng tiêu cực đến người dùng trên MXH. Chương 1 đã trình bày tổng quan về các tác hại mà thông tin sai lệch gây ra đối với người dùng MXH. Chúng ta có thể tóm gọn lại những hậu quả vô cùng to lớn gây ra bởi thông tin sai lệch bằng những nội dung sau: Đối với những vấn đề mang tính xã hội, những thông tin sai lệch ảnh hưởng tiêu cực đến tâm lý, đời sống tinh thần của người dùng khi chúng được phát tán trên mạng. Nó có thể ảnh hưởng đến tinh thần, thái độ, thậm chí cả kinh tế của khu vực người dùng sinh sống. Trong hoạt động kinh doanh, những thông tin sai lệnh tiêu cực về sản phẩm của một doanh nghiêp ảnh hưởng xấu đến tài chính, giá bán, doanh thu, và thậm chí là thương hiệu của doanh nghiệp đó. Đối với từng cá nhân, những thông tin sai lệnh về họ có thể ảnh hưởng rất xấu, làm đảo lộn cuộc sống của họ. 

Những tác hại kể trên cho thấy việc đối phó với các thông tin sai lệch là vô cùng cấp bách. Việc phát hiện ngồn thông tin sai lệch là cơ sở cho các giải pháp ngăn chặn sự phát tán của chúng. Nguồn phát tán thông tin sai lệch có thể được phát hiện thông qua khảo sát người dùng hoặc các phương pháp khai phá dữ liệu. Trong việc giám sát các thông tin sai lệch H.Zhang \cite{zhang1} đã đề xuất giải pháp tìm số node giám sát nhỏ nhất sao cho có thể phát hiện thông tin sai lệch với tỉ lệ $\tau$. 

Để có thể đưa ra giải pháp hiệu quả trong việc ngăn chặn sự lan truyền của thông tin sai lệch, trước tiên chúng ta phải hiểu được cơ chế thông tin sai lệch lan truyền trên MXH. Chương này phân tích quá trình lan truyền thông tin sai lệch dưới hai mô hình lan truyền: Mô hình bậc độc lập và mô hình ngưỡng tuyến tính, đây là hai mô hình phổ biến, được sử dụng rộng rãi trong các công trình nghiên cứu liên quan đến vấn đề lan truyền thông tin, lan truyền ảnh hưởng trên MXH. Đồng thời trong chương này cũng trình bày tổng quan một số hướng nghiên cứu liên quan đến bài toán hạn chế lan truyền thông tin sai lệch trên mạng xã hội trực tuyến.


\section{Mô hình lan truyền thông tin}
Thông tin được phát tán trên các MXH từ người dùng này đến người dùng khác thông qua nhiều hoạt động đăng bài, chia sẻ, bình luận. Kempe \cite{kemple1} là người đầu tiên đưa ra các mô hình phát tán thông tin, trong đó đã đưa ra hai mô hình phát tán thông tin cơ bản là Mô hình tầng độc lập (Independent Cascade – IC) và Mô hình ngưỡng tuyến tính (Linear Threshold – LT). Hai mô hình này sau đó đã được sử dụng rộng rãi trong các bài toán liên quan đến lan truyền, phát tán thông tin.

Các mô hình phát tán thông tin là cơ sở cho việc nghiên cứu hạn chế thông tin sai lệch cũng như các tin đồn thất thiệt. Các nghiên cứu về chủ đề này những năm gần đây đều dựa trên hai mô hình IC, LT và các biến thể của chúng \cite{kemple2}, \cite{Golden}, \cite{Carnes}. Theo đó, một mạng xã hội được biểu diễn bởi các thành phần như sau:
\begin {itemize}
	\item V là tập hợp các đỉnh của đồ thị, |V| = n, biểu diễn những người dùng tồn tại trong MXH.

	\item E $\subset$ VxV hợp các cạnh của đồ thị, |E| = m gồm m cạnh có hướng, biểu diễn mối quan hệ giữa các cá nhân trong MXH.

	\item w(u,v) là trọng số của cạnh (u, v) là một số thực dương biểu diễn cho các tần số tương tác, trao đổi giữa hai người dùng. w(u, v) = 0 nếu giữa hai đỉnh u và v không tồn tại cạnh, w(u, v) > 0 nếu giữa u và v tồn tại cạnh nối.
\end {itemize}
Do G là đồ thị có hướng nên cạnh (u,v) được gọi là cạnh đi ra từ u, cạnh (v,u) được gọi là cạnh đi vào đỉnh u. Ta ký hiệu N$_{out}$(u) và N$_{in}$(u) tương ứng là tập hợp các đỉnh hàng xóm đi ra và đi vào đỉnh u.

Quá trình lan truyền thông tin theo các bước thời gian rời rạc, với thời gian t = 0, 1, 2, ... Gọi S$_{t}$ $\subset$ V là tập các đỉnh ở trạng thái {\itshape kích hoạt} tại thời điểm t. Tập các đỉnh là nguồn phát tán thông tin sai lệch ban đầu hay còn gọi là tập hạt giống, kí hiệu là S$_{0}$

Tại mỗi bước thời gian t, đỉnh u ở trạng thái kích hoạt nếu u $\subset$ S$_{0}$ hoặc u nhận được thông tin sai lệch từ các đỉnh hàng xóm ở trạng thái kích hoạt và chấp nhận thông tin này để tiếp tục chia sẻ, phát tán những thông tin sai lệch đó đến những đỉnh khác trong các bước tiếp theo, quá trình kích hoạt này ở mỗi mô hình lan truyền là khác nhau, ngược lại u ở trạng thái {\itshape không kích hoạt}.

Hiện nay có nhiều mô hình lan truyền thông tin đang được nghiên cứu và sử dụng, tiêu biểu trong số đó là: Mô hình ngưỡng (Threshold Model) \cite{kemple2}, mô hình tầng (Casacading Model) \cite{Golden}, mô hình lan truyền ảnh hưởng cạnh tranh (Competitive Influence Diffusion Model) \cite{Carnes}, mô hình dịch bệnh (Epidemic Model) \cite{leskovec}. Trong nội dung của đề tài, nhóm tác giả sử dụng mô hình ngưỡng tuyến tính (Linear Threshold – LT) và mô hình tầng độc lập (Independent Cascade – IC) \cite{kemple1} để mô tả quá trình lan truyền thông tin sai lệch trên MXH.

Trong nghiên cứu này, các tác giả đã tìm hiểu các mô hình trên qua đó có thể làm rõ tính chất về hành vi lan truyền thông tin của người dùng và chọn ra mô hình phù hợp để áp dụng và cũng đưa ra các mô hình phù hợp hơn với thực tiễn. Sau đây là sự mô tả chi tiết mô hình tiêu biểu được nhiều nghiên cứu sử dụng đó là: Mô hình tầng độc lập (IC) và Mô hình ngưỡng tuyến tính (LT).
	\subsection{Mô hình tầng độc lập}
	Mô hình tầng độc lập (Independent cascade – IC) được đề xuất bởi Kempe \cite{kemple1} dựa trên các mô hình tương tác trong các hệ thống hạt và nghiên cứu về tiếp thị. Mô hình IC có liên quan tới mô hình dịch bệnh (Epidemic models). Đặc trưng chính của mô hình IC là quá trình lan truyền thông tin dọc theo các cạnh của đồ thị một cách độc lập với nhau.
	
	Trong mô hình IC, mỗi cạnh (u,v) $\in$ E được gán một xác suất ảnh hưởng (Influence Probability) $\rho$ $_{uv}$ $\in$ [0,1] biểu diễn mức độ ảnh hưởng của đỉnh u với đỉnh v. Nếu (u,v) $\notin$ E, thì $\rho$ $_{uv}$ = 0.
	
	Quá trình lan truyền thông tin trên mô hình IC diễn ra theo bước thời gian rời rạc, tạo ra tập các đỉnh {\itshape kích hoạt} theo quy tắc sau:
	\begin {itemize}
		\item Tại thời điểm t = 0, tập đỉnh ở trạng thái kích hoạt chính là tập nguồn phát thông tin sai lệch S $_{0}$
	
		\item Tại thời điểm t  1, đầu tiên ta gán S$_{t}$ bằng S$_{t-1}$ sau đó với mỗi nút v $\notin$ S$_{t-1}$, và với mỗi nút u $\in$ N$_{in}$(v) $\bigcap$ (S$_{t-1}$ \ S$_{t-2}$), u thực hiện một lần thử kích hoạt bằng cách áp dụng phép thử Bernoulli (Phép tung đồng xu độc lập) với xác suất thành công là p(u,v). Nếu thành công ta thêm v vào tập S$_{t}$ và nói rằng u kích hoạt v tại thời điểm t. Nếu nhiều nút kích hoạt v tại thời điểm t, kết quả tương tự xảy ra, v được thêm vào tập S$_{t}$. Nói cách khác, sau khi nút u được kích hoạt tại thời điểm t-1, ngay lập tức trong thời điểm t, u có {\bfseries một cơ hội duy nhất} để kích hoạt các hàng xóm chưa được kích hoạt v của nó với xác suất p(u,v), và những sự kích hoạt này là độc lập với nhau. Nếu nút u không kích hoạt v tại thời điểm t, nó sẽ không thử kích hoạt v tại các thời điểm sau nữa. Và một khi một nút đã được kích hoạt, nó vẫn giữ trạng thái đó ở các bước sau.
		
		\item Nếu tại thời điểm t, không có nút nào được kích hoạt thêm nữa, nghĩa là S$_{t}$ = S$_{t-1}$, tập các nút {\itshape kích hoạt} sẽ không còn thay đổi nữa, và quá trình truyền tin kết thúc với tập các nút bị kích hoạt cuối cùng là S$_{t}$
	\end {itemize}
	Hình \ref{refhinh2_1} chỉ ra một ví dụ của quá trình lan truyền thông tin trên mô hình IC. Các đỉnh màu da cam và màu xanh tương ứng biểu diễn các đỉnh ở trạng thái kích hoạt, và không kích hoạt. Cạnh liền màu đỏ từ u đến v biểu diễn u kích hoạt thành công v, cạnh nét đứt màu xanh từ u đến v biểu diễn u kích hoạt không thành công v.
		\begin{center}
			\begin{figure}[htp]
				\begin{center}
					\includegraphics [scale=.5]{picture/Hinh2_1}
				\end{center}
				\caption{Một số ví dụ quá trình lan truyền thông tin trên mô hình IC}
				\label{refhinh2_1}
			\end{figure}
		\end{center}
	Tại thời điểm bắt đầu t=0, hai đỉnh v$_{1}$, v$_{2}$ ở trạng thái {\itshape kích hoạt}. Ở thời điểm t = 1, v$_{1}$ kích hoạt thành công v$_{5}$ nhưng thất bại với v$_{3}$, trong khi đó v$_{2}$ kích hoạt thành công v$_{3}$ và v$_{4}$ nhưng thất bại với v$_{6}$. Tại thời điểm t = 2, v$_{3}$ kích hoạt thất bại v$_{6}$ trong khi v$_{5}$ kích hoạt thành công v$_{6}$ nhưng thất bại với v$_{9}$. Ở bước t = 3, v$_{6}$ kích hoạt thất bại v$_{7}$, đến đây quá trình lan truyền thông tin kết thúc do không có đỉnh nào được kích hoạt thêm.
	
	Mô hình IC phù hợp cho quá trình lan truyền thông tin hoặc virus, đó là các môi trường mà việc tiếp xúc với một nguồn là đủ để một cá nhân được kích hoạt.
	
	\subsection{Mô hình ngưỡng tuyến tính}
	Mô hình IC phù hợp để mô tả sự lan truyền thông tin đơn giản, ở đó một đỉnh có thể được kích hoạt từ một đỉnh duy nhất. Tuy nhiên trong thực tế có nhiều trường hợp một cá nhân cần nhiều sự tác động của các cá nhân khác để thay đổi hành vi của mình. Có thể kể đến các trường hợp như khi người dùng tiếp nhận một thông tin mới, một công nghệ mới, hay một thông tin sai lệch bôi xấu danh dự của các đồng chí lãnh đạo Đảng và Nhà nước, người dùng MXH cần được củng cố tích cực từ nhiều nguồn độc lập trong số bạn bè và người quen của họ trước khi họ thay đổi suy nghĩ của mình, chấp nhận thông tin đó.
	
	Các nhà khoa học đã đề xuất khái niệm {\itshape hành vi ngưỡng} để mô tả các hành vi kiểu như trên. Khi một hàm tổng hợp của các người dùng đã kích hoạt trên mạng đạt đến một ngưỡng nhất định thì  đối tượng sẽ được kích hoạt, xét đến hành vi ngưỡng mà mỗi cá nhân chỉ được kích hoạt khi tiếp nhận ảnh hưởng từ nhiều hơn hai nguồn thông tin.
	
	Mô hình ngưỡng tuyến tính (Linear Threshold – LT) là mô hình khuếch tán ngẫu nhiên được đề xuất bởi Kempe \cite{kemple1}. Trong mô hình LT, mỗi cạnh (u,v) $\in$ [0,1] biểu diễn mức độ ảnh hưởng của đỉnh u đến đỉnh v. Nếu (u,v) $\notin$ E thì w(u,v)=0. Các trọng số này được chuẩn hóa sao cho với mỗi đỉnh v, tổng trọng số tất cả các cạnh đi đến đỉnh v lớn nhất bằng 1, tức là: $\sum$ $_{u}$ w(u,v) $\leq$ 1 %$\in$ N$_{in}$(u) 
	
	Tùy vào đặc tính của từng người dùng tương ứng, mỗi đỉnh v $\in$ V có một giá trị $\theta$$_{v}$ $\in$ [0,1], biểu diễn ngưỡng đỉnh v bị ảnh hưởng bởi các đỉnh kích hoạt hàng xóm. Quá trình lan truyền thông tin trên mô hình LT diễn ra theo bước thời gian rời rạc, tạo ra tập các đỉnh kích hoạt theo quy tắc sau:
	\begin {itemize}
		\item Tại thời điểm t = 0, tập đỉnh ở trạng thái kích hoạt chính là tập nguồn phát thông tin sai lệch S$_{0}$.
	
		\item Tại thời điểm t  1, đầu tiên ta gán S$_{t}$ bằng S$_{t-1}$. Sau đó với mỗi đỉnh chưa được kích hoạt v $\in$ V \ S$_{t-1}$, nếu tổng ảnh hưởng từ những đính hàng xóm kích hoạt tới v vượt ngưỡng $\theta$$_{0}$, tức là % $\sum$ $_{u}$ w(u,v) $\geqslant$ $\theta$ $_{0}$ thì đỉnh v được kích hoạt, ta thêm v vào tập S$_{t}$.
	
		\item Nếu tại thời điểm t, không có nút nào được kích hoạt thêm nữa, nghĩa là S$_{t}$ = S$_{t-1}$, tập các nút kích hoạt sẽ không còn thay đổi nữa, và quá trình truyền tin kết thúc với tập các nút bị kích hoạt cuối cùng là S$_{t}$.
	\end {itemize}
	Sự ngẫu nhiên trong việc lựa chọn ngưỡng $\theta$$_{0}$ từ 0 đến 1 phản ánh sự thiếu thông tin về ngưỡng nội bộ của mỗi cá nhân. Điều này phản ánh khá đúng với thực tế xã hội, bởi vì sự chập nhận thông tin của mỗi người, tại những thời điểm khác nhau là khác nhau, và rất khó để nắm bắt. 
	
	Hình \ref{refhinh2_2} chỉ ra một ví dụ quá trình lan truyền thông tin trên mô hình LT. Các đỉnh màu da cam và màu xanh tương ứng biểu diễn các đỉnh ở trạng thái {\itshape kích hoạt}, và {\itshape không kích hoạt}. Các cạnh liền màu đỏ cùng đến đỉnh v biểu diễn các cạnh này đồng thời cố thử kích hoạt đỉnh v và thành công.
	
	\begin{center}
		\begin{figure}[htp]
			\begin{center}
				\includegraphics [scale=.75]{picture/Hinh2_2}
			\end{center}
			\caption{Ví dụ quá trình lan truyền trên mô hình LT}
			\label{refhinh2_2}
		\end{figure}
	\end{center}	
	Tại thời điểm t=0, tất cả các đỉnh được khởi tạo ngẫu nhiên ngưỡng $\theta$$_{0}$ $\in$ [0,1], hai đỉnh v$_{1}$ và v$_{2}$ là các đỉnh hạt giống. Ở thời điểm t = 1, v$_{1}$ và v$_{2}$ kích hoạt thành công v$_{3}$, v$_{1}$ cũng phải cách hoạt thành công v$_{5}$ và v$_{2}$ kích hoạt thành công v$_{4}$; tuy nhiên v$_{6}$ lại không kích hoạt thành công vì tổng trọng số các cạnh đi đến v$_{6}$ là 0.3, trong khi ngưỡng kích hoạt của v$_{6}$ 0.7. Tại thười điểm t = 2, các đỉnh hàng xóm đi đến v$_{6}$ là v$_{2}$, v$_{3}$, v$_{5}$ đã được kích hoạt cho nên tổng trọng số các cạnh đi đến là 0.7 đủ để kích hoạt v$_{6}$. Tại bước t = 3, quá trình lan truyền thông tin kết thúc do không có đỉnh nào được kích hoạt thêm.

\section{Một số hướng nghiên cứu liên quan bài toán hạn chế lan truyền thông tin sai lệch trên mạng xã hội}
Tối ưu hóa ảnh hưởng của các đối tượng trên MXH là bài toán liên quan đến lan truyền thông tin trên MXH được nghiên cứu lần đầu tiên bởi Domingos và Richardson, 2001 \cite{pedro}. Sau khi Kempe, 2003 \cite{kemple1} lần đầu tiên xây dựng vấn đề tối ưu hóa ảnh hưởng trên MXH theo cách tối ưu hóa rời rạc, có rất nhiều nghiên cứu liên quan tập trung giải quyết bài toán tối đa hóa ảnh hưởng của thông tin trên mạng xã hội như nghiên cứu của Leskovec, 2007 \cite{leskovec}, Goyal, 2011 \cite{goyal}, Wei Chen, 2014 \cite{chen}.

Bên cạnh vấn đề lan truyền thông tin, lan truyền ảnh hưởng cũng có rất nhiều nghiên cứu tập trung giải quyết bài toán hạn chế thông tin sai lệch lan truyền trên các MXH trực tuyến. Một số nghiên cứu tập trung vào việc nhận dạng thông tin sai lệch và tin đồn (Rumor) dựa trên đặc trưng ngôn ngữ, cấu trúc, thời gian như nghiên cứu của Qazvinian, 2011, \cite{qazvin} và Kwon, 2013, \cite{kwon}.

Một số khác, nghiên cứu vấn đề xác định tập đỉnh là nguồn phát thông tin sai lệch ban đầu. Chẳng hạn, Dung T. Nguyen và các cộng sự, 2012, \cite{nguyen1} đã nghiên cứu bài toán xác định k nguồn phát tán thông tin sai lệch khả nghi nhất từ tập người dùng bị kích hoạt bởi thông tin sai lệch cho trước và chứng minh bài toán thuộc lớp NP-khó xét trên mô hình lan truyền IC, đồng thời nhóm tác giả đã đề xuất hai thuật toán dựa trên cách tiếp cận xếp hạng (Ranking) và cách tiếp cận xấp xỉ đạt tỉ lệ tối ưu 1 - $\dfrac{1}{e}$ - $\varepsilon$

Trong việc hạn chế phát tán thông tin sai lệnh, Zhang \cite{zhang1} để xuất bài toán tìm số nút nhỏ nhất trong khoảng cách $\delta$ (vùng N$_{}$ ()) đối với nguồn phát thông tin sai lệnh để vô hiệu hóa sao cho thông tin sai lệnh đến nút đích r được giới hạn. Bên cạnh đó, một số tác giả đề xuất giải pháp hạn chế sự lan truyền thông tin sai lệch trên mạng xã hội bằng cách chọn ra một số đỉnh ban đầu để tiêm thông tin tốt, từ đó lan truyền những thông tin này trên cùng mạng nhằm thuyết phục những người dùng khác tin theo, trong đó sử dụng các mô hình lan truyền thông tin khác nhau \cite{nguyen9}, \cite{nguyen30}.

Trong việc khử nhiễm đối với nguồn tin sai lệnh, Nguyen \cite{nguyen9} đề xuất bài toán tìm tập người dùng hạt giống sao cho tỷ lệ khử nhiễm sau thời gian T đối với nguồn thông tin sai lệnh I trong mạng là $\beta$ $\in$ (0,1). Tức là khử nhiễm đối với các nguồn phát này sao cho số người dùng được khử nhiễm là $\beta$.|V|. Ceren \cite{ceren} đưa ra bài toán phản bác lại thông tin sai lệnh bằng cách chọn k người dùng để thuyết phục họ nhận thức được các thông tin để phản bác lại, triệt tiêu các thông tin sai lệnh. Trong nghiên cứu này, nhóm tác giả đã xây dựng mô hình Oblivious Independent Campaign, họ chứng minh đây là bài toán NP-Khó và hàm mực tiêu là hàm đơn điệu tăng và submodular.

Liên quan gần nhất đến vấn đề nghiên cứu trong đề tài của nhóm tác giả là công trình nghiên cứu của H. Zhang, 2016 \cite{zhang31}. Trong nghiên cứu của mình, H. Zhang đề xuất hai bài toán: Bài toán phát hiện thông tin sai lệch yêu cầu xác định k vị trí đặt giám sát (Monitor) trên MXH sao cho cực đại hóa xác suất phát hiện thông tin sai lệch và Bài toán đặt giám sát yêu cầu tìm ra tập đỉnh có kích thước nhỏ nhất để đặt giám sát sao cho xác suất thông tin sai lệch kích hoạt thành công đỉnh r nhỏ hơn một ngưỡng cho trước.
