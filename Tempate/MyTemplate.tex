\documentclass[14pt,oneside,titlepage,reqno,a4paper]{mybook}%{book}%%{these_gi}%twosidedraft,
\usepackage{xkeyval}
\usepackage{titlesec}
\usepackage{times} 
\usepackage{amsmath} 
% Định nghĩa thuật toán
%Tạo Index
\usepackage{algorithm}
\usepackage{algorithmic}
\usepackage{multirow}
\usepackage{makeidx}
%Thụt vào đầu dòng mỗi đoạn
\usepackage{indentfirst}
\setlength{\parindent}{1cm} 
%\input setbmp
%Để chế độ thụt vào đầu dòng trong chế độ liệt kê số (enumerate)
 \usepackage{enumitem} 
 \usepackage{enumerate}
% \usepackage{lipsum} 
 \setlist[enumerate]{itemindent=\dimexpr\labelwidth+\labelsep\relax,leftmargin=1em} 
% \setlist[itemize][label =-]{itemindent=\dimexpr\labelwidth+\labelsep\relax,leftmargin=1em} 
\setlist[itemize,1]{label=-}%thay vì dấu chấm thì là dấu gạch đầu dòng
 %
 %Thụt vào đầu dòng đối với tiêu đề của section


\setcounter{secnumdepth}{5} %Đặt chiều sâu của các section. Với số 5, ta có đến paragraph
\renewcommand\thesection{\bfseries \arabic{section}.}
%\renewcommand\thesection{\Roman{section}.} %Đặt số thứ tự của Section là I, II, III
% Cài đặt độ sâu cho tiêu đề: Ở đây là chương rồi đến mục
\renewcommand\thesubsection{\arabic{section}.\arabic{subsection}.} %Đặt số thứ tự của subseciton là 1., 2., 
\renewcommand\thesubsubsection{\alph{subsubsection})} %Số thứ tự của subsubsection là a), b)
%
%\titleformat{\section}[runin]{\bfseries}{\thesection}{1em}{} %Cho phép chữ đi ngay sau tiêu đề. Cái này không dùng ở đây.
\titleformat{\section}[block]{\normalfont}{\thesection}{.5em}{ \bfseries}
% Đặt cỡ chữ cho tiêu đề : \large: cớ lớn; \bfseries: là đậm
%\titleformat{\section}[block]{\normalfont}{\thesection}{.5em}{\MakeUppercase}% Cài đặt form của section\bfseries. Makeuppercase là để chữ in hoa cho section
\titlespacing{\section}{2em}{.5em}{.5em}%Khoảng cách section đến đầu dòng: thụt vào 2em = 1cm
\titleformat{\subsection}[block]{\bfseries}{\thesubsection}{.5em}{}%Tương tự section\itshape
\titlespacing{\subsection}{2em}{.5em}{.5em}
%==============Hằng thêm vào
\titleformat{\subsubsection}[block]{\bfseries}{\thesubsection}{.5em}{}%Tương tự section\itshape
\titlespacing{\subsubsection}{2em}{.5em}{.5em}
%=====================
%Chú ý: \bfseries là chữ đậm, \itshape là chữ nghiêng.
\titleformat{\subsubsection}[block]{\bfseries\itshape}{\thesubsubsection}{.5em}{}
\titlespacing{\subsection}{2em}{.5em}{.5em}
%

%\usepackage[absolute]{textpos} 
%\usepackage{cases}
\usepackage{hyperref}		%Tạo liên kết từ mục lục tới các phần tương ứng
\usepackage[mathscr]{eucal}%
\usepackage[nottoc]{tocbibind}

%\RequirePackage{refcount}[2006/02/12]
%\usepackage{pstricks}

\usepackage[utf8]{vietnam}
%\documentclass[a4paper, 14pt]{report}
%\usepackage[utf8]{vntex}
%\input{macros}
\usepackage{scalefnt}
\usepackage[left=3.0 cm, right=2 cm, top=2.5 cm, bottom=2.5 cm]{geometry}%tạo lề của trangleft=3.5 cm, right=2.5 cm, top=2.5 cm, bottom=2.5 cm]
\usepackage{graphicx}
\usepackage{amsmath,amsthm,array}%
\usepackage{slashbox}
%==================
% Fancy style
%======================
\usepackage{fancyhdr}
\pagestyle{fancy}

%\renewcommand{\chaptermark}[1]{\markboth{#1}{}}
\renewcommand{\sectionmark}[1]{}%\markright{\Roman{section}.\ #1}}

\renewcommand{\headrulewidth}{0.4pt}%Tạo bề dày của các đường thẳng chia header, footer
%\renewcommand{\footrulewidth}{0pt}
\fancyhead{}
\fancyfoot{}
\fancyhead[RO,LE]{\normalsize\itshape{{\nouppercase{\rightmark}}}}
\fancyhead[LO,RE]{\normalsize\itshape{{\nouppercase{\leftmark}}}}
\fancyfoot[C]{{\normalsize{\thepage}}}
%\fancyfoot[LO,LE]{\textbf{{\normalsize{Lý thuyết thông tin - Phạm Văn Cảnh, Phạm Thị Hằng, Lê Huy Dũng}}}}
%=======Các Định lý, Định nghĩa, Bổ đề, Hệ quả =======

 
%\newtheorem{vd}{Ví dụ}[section]
\newtheorem{theo}{Định lý}[chapter]
\newtheorem{lem}{Bổ đề}[chapter]%[section]
\newtheorem{coro}{Hệ quả}[chapter]%[section]
\theoremstyle{definition}
\newtheorem{dn}{Định nghĩa}[chapter]%[section]
\newtheorem{define}{Định nghĩa}[chapter]
\floatstyle{ruled}
\newfloat{algorithm}{htbp}{loa}
\floatname{algorithm}{Thuật toán}

\renewcommand{\thedn}{\arabic{chapter}.\arabic{dn}}%
%\newenvironment{proof}[1][Chứng minh.]{\begin{trivlist}
%\item[\hskip \labelsep {\bfseries#1}]}{\end{trivlist}}
\newenvironment{vd}[1][Ví dụ.]{\begin{trivlist}
\item[\hskip \labelsep {\bfseries#1}]}{\end{trivlist}}

\newtheorem{example}{Ví dụ}[chapter]
\newenvironment{remark}[1][Chú ý.]{\begin{trivlist}
		\item[\hskip \labelsep {\bfseries#1}]}{\end{trivlist}}

%Khoảng cách giữa các đoạn, cách dòng, giãn dòng
\setlength{\baselineskip}{5truept} 
%\renewcommand{\baselinestretch}{1.3}%Cách dòng 1.2
\linespread{1.3}
%\abovedisplayskip =6pt plus 6pt minus 9pt
%\abovedisplayshortskip =0pt plus 6pt minus 9pt
\belowdisplayskip =\abovedisplayskip
\belowdisplayshortskip =\abovedisplayshortskip


%\scalefont{1.4}
%\fontsize{13pt}{20pt}\selectfont
%\fontsize{13pt}{22pt}{\selectfont}
%\parindent=0pt
%=================Đánh số cho Chương, Mục, Định lí, Định nghĩa...==========
%
\renewcommand{\thetheo}{\arabic{chapter}.\arabic{theo}}%
\renewcommand{\thecoro}{\arabic{chapter}.\arabic{coro}}%
%\renewcommand{\thelemma}{\arabic{chapter}.\arabic{section}.\arabic{lemma}}%\arabic{section}.
%\renewcommand{\theremark}{\arabic{chapter}.\arabic{section}.\arabic{remark}}%\arabic{section}.
%\renewcommand{\theproposition}{\arabic{chapter}.\arabic{section}.\arabic{proposition}}%
\renewcommand{\theequation}{\arabic{chapter}.\arabic{equation}}%.\arabic{section}.
%\renewcommand{\thecorollary}{\arabic{chapter}.\arabic{section}.\arabic{corollary}}%
%\renewcommand{\thenote}{\arabic{chapter}.\arabic{section}.\arabic{note}}%
%\renewcommand{\theexample}{\arabic{chapter}.\arabic{section}.\arabic{example}}%
%\renewcommand{\thedefine}{\arabic{chapter}.\arabic{section}.\arabic{define}}%
%\renewcommand{\theproperty}{\arabic{chapter}.\arabic{section}.\arabic{property}}%
%\renewcommand{\thehypothesis}{\arabic{chapter}.\arabic{section}.\arabic{hypothesis}}%
%\renewcommand{\thefootnote}{\ensuremath{(\arabic{footnote})}}%
%\renewcommand{\theproblem}{\arabic{problem}}%
%\renewcommand{\thesubsubsection}{\alph{subsubsection}}%

% .........Annotation.......
% Math Screen Lettre
\def\A{{\mathscr A}}
\def\B{{\mathscr B}}
\def\C{{\mathscr C}}
\def\D{{\mathscr D}}
\def\E{{\mathscr E}}
\def\F{{\mathscr F}}
\def\G{{\mathscr G}}
\def\H{{\mathscr H}}
\def\K{{\mathscr K}}
\def\L{{\mathscr L}}
\def\M{{\mathscr M}}
\def\N{{\mathscr N}}
\def\O{{\mathscr O}}
\def\P{{\mathscr P}}
\def\Q{{\mathscr Q}}
\def\R{{\mathscr R}}
\def\SS{{\mathscr S}}
\def\T{{\mathscr T}}
\def\U{{\mathscr U}}

%Caligraph

\def\cA{{\mathcal A}}
\def\cB{{\mathcal B}}
\def\cC{{\mathcal C}}
\def\cD{{\mathcal D}}
\def\cE{{\mathcal E}}
\def\cF{{\mathcal F}}
\def\cG{{\mathcal G}}
\def\cH{{\mathcal H}}
\def\cK{{\mathcal K}}
\def\cL{{\mathcal L}}
\def\cM{{\mathcal M}}
\def\cN{{\mathcal N}}
\def\cO{{\mathcal O}}
\def\cP{{\mathcal P}}
\def\cQ{{\mathcal Q}}
\def\cR{{\mathcal R}}
\def\cS{{\mathcal S}}
\def\cT{{\mathcal T}}
\def\cU{{\mathcal U}}
\def\cX{{\mathcal X}}
\def\cY{{\mathcal Y}}
\def\cZ{{\mathcal Z}}
% Bold type
\def\one{{\mathbf 1}}
\def\bA{{\mathbf A}}
\def\bB{{\mathbf B}}
\def\bC{{\mathbf C}}
\def\bD{{\mathbf D}}
\def\bE{{\mathbf E}}
\def\bF{{\mathbf F}}
\def\bG{{\mathbf G}}
\def\bH{{\mathbf H}}
\def\bI{{\mathbf I}}
\def\bJ{{\mathbf J}}
\def\bK{{\mathbf K}}
\def\bL{{\mathbf L}}
\def\bM{{\mathbf M}}
\def\bN{{\mathbf N}}
\def\bO{{\mathbf O}}
\def\bP{{\mathbf P}}
\def\bQ{{\mathbf Q}}
\def\bR{{\mathbf R}}
\def\bS{{\mathbf S}}
\def\bT{{\mathbf T}}
\def\bU{{\mathbf U}}
\def\bV{{\mathbf V}}
\def\bX{{\mathbf X}}
\def\bY{{\mathbf Y}}
\def\bZ{{\mathbf Z}}

\def\ba{{\mathbf a}}
\def\bb{{\mathbf b}}
\def\bc{{\mathbf c}}
\def\bbd{{\mathbf d}}
\def\bbe{{\mathbf e}}
\def\bbf{{\mathbf f}}
\def\bbg{{\mathbf g}}
\def\bbh{{\mathbf h}}
\def\bbi{{\mathbf i}}
\def\bbj{{\mathbf j}}
\def\bbk{{\mathbf k}}
\def\bbl{{\mathbf l}}
\def\bbm{{\mathbf m}}
\def\bbn{{\mathbf n}}
\def\bbo{{\mathbf o}}
\def\bbp{{\mathbf p}}
\def\bbq{{\mathbf q}}
\def\bbr{{\mathbf r}}
\def\bbs{{\mathbf s}}
\def\bbt{{\mathbf t}}
\def\bbu{{\mathbf u}}
\def\bbv{{\mathbf v}}
\def\bbx{{\mathbf x}}
\def\bby{{\mathbf y}}
\def\bbw{{\mathbf w}}
\def\bbz{{\mathbf z}}

%=========Blackboard bold math alphabets========= 
%=========[used for set notation]=========
%
\def\AA{{\mathbb A}}
\def\BB{{\mathbb B}}
\def\CC{{\mathbb C}}
\def\EE{{\mathbb E}}
\def\II{{\mathbb I}}
\def\NN{{\mathbb N}}
\def\OO{{\mathbb O}}
\def\PP{{\mathbb P}}
\def\QQ{{\mathbb Q}}
\def\RR{{\mathbb R}}
\def\WW{{\mathbb W}}
\def\XX{{\mathbb X}}
\def\YY{{\mathbb Y}}
\def\ZZ{{\mathbb Z}}
\def\sS{{\mathbb S}}

%================DeclareMathOperator=========
%=============================================
%  Bỏ trang trắng ở trang chẵn
%=============================================
\makeatletter
\def\cleardoublepage{\clearpage\if@twoside%
    \ifodd\c@page\else
    \vspace*{0cm}
    \hfill
    \begin{center}
        \emph{Trang này được để trắng} 
    \end{center}
    \vspace{\fill}
    \thispagestyle{empty}
    \newpage
    \if@twocolumn\hbox{}\newpage\fi\fi\fi
}
\makeatother
