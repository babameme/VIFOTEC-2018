\chapter*{Kết Luận}
Sự bùng nổ của thông tin sai lệch trên các mạng xã hội đang là một nguy cơ lớn đối với người dùng. Sự lan truyền nhanh chóng của thông tin sai lệch đến từ sự chủ quan của người dùng hoặc được lan truyền một cách có chủ đích với động cơ, mục đích không trong sáng. Hơn nữa, hoạt động này đem lại những hậu quả nghiêm trọng đối với cá nhân, tổ chức không chỉ về kinh tế chính trị mà còn tác động đến tâm lý, cuộc sống con người, gây mất ổn định an ninh quốc gia. Do đó, việc đưa ra giải pháp hiệu quả nhằm ngăn chặn kịp thời sự lan truyền của thông tin sai lệch trên mạng xã hội trực tuyến là viêc làm hết sức cấp thiết nhằm giảm thiểu tối đa thiệt hại do thông tin sai lệch gây ra đối với cá nhân, tổ chức, góp phần làm trong sạch môi trường mạng, đồng thời nâng cao sự tin tưởng của người dùng đối với những thông tin trên mạng xã hội.

Trong đề tài này, nhóm tác giả đề xuất hai kịch bản ngăn chặn có tính ứng dụng trong thực tế, tương ứng là hai bài toán tối ưu. Dựa trên 2 mô trình lan truyền thông tin thường được sử dụng là mô hình ngưỡng tuyến tính (Linear Threshold Model), mô hình bậc độc lập (Independent Casade Model), nhóm tác giả đề xuất mô hình ngưỡng tuyến tính với bước thời gian rời rạc (Time – Discrete Linear Threshold Model) từ đó đưa ra hai giải pháp ngăn chặn hiệu quả. Đề tài đã đạt được một số kết quả chính như sau:

\begin{itemize}
	\item Tìm hiểu tổng quan về mạng xã hội, các đặc trưng cơ bản của mạng xã hội, lợi ích của mạng xã hội, các nguy cơ lây lan và tác hại của thông tin sai lệch trên mạng xã hội.
	
	\item Tìm hiểu cơ chế lan truyền thông tin và đặc tính của hai mô hình lan truyền thông tin thường được sử dụng, từ đó đề xuất một mô hình lan truyền mới, phù hợp với bài toán khảo sát.
	
	\item Đề xuất hai kịch bản, hai bài toán tương ứng là LSE và TMB, chứng minh hai bài toán này thuộc lớp \#P – Khó, đề xuất hai thuật toán hiệu quả FLE và STMB với tốc độ thực thi nhanh và hiệu quả tốt.
	
	\item Thu thập dữ liệu thực từ mạng xã hội Facebook với phạm vi xung quanh các đối tượng cầm đầu, chủ mưu các tổ chức phản động, đối tượng là người có uy tín lớn. Từ đó áp dụng giải pháp FLE và STMB để đưa ra tập người dùng tiềm năng để vô hiệu hóa hoạt động của họ và đem lại hiệu quả tốt nhất.
	
\end{itemize}

\textbf{Hướng phát triển - kiến nghị} 

Trong thời gian tới, nhóm tác giả đề xuất một số hướng phát triển của đề tài như sau:

\begin{itemize}
	\item Thiết kế thuật toán đạt tỉ lệ tối ưu cao hơn, thời gian chạy thuật toán nhanh hơn cho bài toán hạn chế ảnh hưởng của thông tin sai lệch trên mạng xã hội.
	
	\item Mở rộng bài toán hạn chế thông tin sai lệch đối với nhiều nguồn thông tin sai lệch cùng tác động lên mạng.
	
	\item Thiết lập một phần mềm với giao diện trực quan đóng gói tất cả các công cụ để cho người dùng có thể sử dụng dễ dàng.
	
	\item Tìm hiểu các giải pháp thu thập dữ liệu hiệu quả hơn, thực thi với tốc độ nhanh hơn.
	
	\item Sau khi Luật An ninh mạng được thông qua, đề xuất các cơ quan chức năng nghiên cứu, mở rộng và áp dụng những thành tựu, kết quả, giải pháp mà nhóm tác giả đã đề xuất.
\end{itemize}
