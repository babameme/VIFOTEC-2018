\chapter*{MỞ ĐẦU}
%
\addcontentsline{toc}{chapter}{Lời nói đầu}
\markboth{Lời nói đầu}{}
%
Cùng với sự phát triển của công nghệ truyền thôn và mạng internet, các mạng xã hội đã phát triển mạnh mẽ và trở thành một xu hướng mới trên toàn thế giới. Theo những khảo sát gần đây, có gần một nửa dân số thế giới, tức là hơn 3 tỷ người sử dụng mạng xã hội. Nhờ có mạng xã hội, người dùng có thể trao đổi thông tin với nhau một cách nhanh chóng bất kể khoảng cách về địa lý và thời gian. Bên cạnh đó, mạng xã hội còn cung cấp cho người dùng rất nhiều ứng dụng hữu ích, làm cho cuộc sống của con người ngày càng trở nên thuận tiện hơn. Ngoài những đặc tính kế thừa của mạng lưới xã hội thực như: Tương tác giữa người dùng, lan truyền thông tin, tạo ảnh hưởng trong cộng đồng thì mạng xã hội còn mang nhiều đặc tính mới như: Cập nhật thông tin thực lên mạng xã hội một cách nhanh chóng, sự lan truyền thông tin giữa người dùng xảy ra trong thời gian ngắn, sự bùng nổ thông tin với các nguồn tin tức khác nhau, v.v... Có thể nói, hiện nay mạng xã hội đang từng bước trở thành một kho tri thức mới mà con người có thể dễ dang tiếp cận.

Trong bối cảnh đó, các chủ đề nghiên cứu về mạng xã hội