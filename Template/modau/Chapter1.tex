\chapter*{LỜI NÓI ĐẦU}
Thực hiện kế hoạch của Học viện về việc đổi mới chương trình, nội dung đào tạo ngành công nghệ thông tin và an toàn thông tin đáp ứng yêu cầu thực tiễn và bắt kịp sự phát triển của khoa học và công nghệ trong thời kỳ mới. Năm 2016, Khoa Công nghệ và An ninh thông tin đã tiến hành rà soát và tiến hành chỉnh sửa, cập nhật mới một số chương trình học phần trong các chương trình đào tạo do Khoa phụ trách. Lý thuyết thông tin là một trong số các học phần cần chỉnh sửa, cập nhật mới để đảm bảo phù hợp hơn với mục tiêu, chương trình đào tạo, đảm bảo chuẩn kiến thức, kỹ năng và chuẩn đầu ra đã ban hành, đảm bảo cân đối giữa lý thuyết và thực hành, phù hợp với thực tiễn và cập nhật những tri thức mới của ngành Công nghệ thông tin và An toàn thông tin. 
Học phần lý thuyết thông tin là môn học thuộc nhóm các môn cơ sở ngành của ngành An toàn thông tin, có quan hệ chặt chẽ với học phần Xác xuất thống kê và quá trình ngẫu nhiên. Các kiến thức cơ bản trong lý thuyết truyền tin, nén thông tin, bao gồm: Độ đo lượng tin, Entropy, Entropy hợp, có điều kiện, thông tin tương hỗ, mối liên hệ giữa các đại lượng đo lượng tin; nén dữ liệu; kiến thức về mã hoá kênh; các quá trình tuyền tin. 
Hiện nay trong nước chưa có tài liệu nào về môn học này giành cho ngành Công nghệ thông tin và An toàn thông tin, phần lớn các giáo trình gần với môn học đều ở các ngành học khác như: điện, điện tử viễn thông vv.. 
Giáo trình được sử dụng cho môn học Lý thuyết thông tin giành cho ngành Công nghệ thông tin ở các trường đại học lớn trong nước thường viết bằng tiếng anh, trong đó có cuốn Elements of Information Theory của nhà xất bản Wiley thường được sử dụng. Đối với các lớp chuyên ngành của Khoa vấn sử dụng tài liệu này làm tài liệu tham khảo chính, ngoài ra còn cuốn tập bài giảng môn lý thuyết thông tin được nhóm tác giả Phạm Văn Cảnh, Phạm Thị Hằng biên soạn đã được nghiệm thu cấp khoa. 

Với thực trạng trên, tập bài giảng cấp Học viện môn Lý thuyết thông tin được đề xuất biên soạn nhằm mục đích: nâng cấp cuốn tập bài giảng đã có; bổ sung, tham khảo các tài liệu có giá trị trong và ngoài nước; bám sát chương trình khung của môn học. Cuốn tập bài giảng mới biên soạn sẽ là nguồn tài liệu thống nhất phục vụ giảng dạy và học tập học phần Lý thuyết thông tin và là tiền để để nhóm tác giả tiếp tục biên soạn cuốn Giáo trình Lý thuyết thông tin sau này.
